\documentclass[11pt,fleqn]{article}
% \usepackage{cs70,latexsym,epsf}
\usepackage{latexsym,epsf,fleqn}
\usepackage{amsmath,amsthm,amsfonts,amssymb}
\usepackage{mathtools}
\usepackage{array}

\usepackage{geometry}
\geometry{
  a4paper,
  total={170mm,257mm},
  left=20mm,
  top=20mm,
}

\newcommand\Set[2]{\{\,#1\mid#2\,\}}
\newcommand\underoverset[3]{\underset{#1}{\overset{#2}{#3}}}
\newcommand{\mbf}[1]{\mbox{{\bfseries #1}}}
\newcommand{\N}{\mathbb{N}}
\newcommand{\Z}{\mathbb{Z}}
\newcommand{\R}{\mathbb{R}}
\newcommand{\Q}{\mathbb{Q}}

\begin{document}

\section*{CS 70 homework 1 solutions}

Your full name: Joey Yandle
\newline
Your login name: dragon
\newline
Homework 1
\newline
Your section number: 0
\newline
Your list of partners: Galina Vinnik
\newline

\begin{enumerate}
\item
For each of the following, define proposition
symbols for each simple proposition in the argument (for example, $P$ =
``I will ace this homework''). Then write out the logical form of
the argument. If the argument form corresponds to a known inference
rule, say which it is. If not, show that the proof is correct using
truth tables.
\begin{enumerate}
\item  I will ace this homework and I will have fun doing it.
Therefore, I will ace this homework.
\begin{align}
  P &= \text{``I will ace this homework''}\\
  Q &= \text{``I will have fun doing it''}\\
  P &\land Q \implies P
\end{align}

$P \land Q$ implies both $P$ and $Q$, and $P \implies P$.
\begin{align}
\therefore P \land Q \implies P\qed
\end{align}

\item It is hotter than 100 degrees today or the pollution is
dangerous. It is less than 100 degrees today. Therefore, the pollution
is dangerous.
\begin{align}
  P &= \text{``It is hotter than 100 degrees today''}\\
  Q &= \text{``The pollution is dangerous''}\\
  \neg P &= \text{``It is less than 100 degrees today''}\\
  Q &= \text{``The pollution is dangerous''}\\
  (P &\lor Q) \land \neg P \implies Q
\end{align}

We know that not only is $(P \lor Q)$ true, but also $P$ is not true, so $Q$ must be true.
\begin{align}
\therefore (P \lor Q) \land \neg P \implies Q \qed
\end{align}

(This is ignoring the possibility of the temperature being exactly 100 degrees, in which case it is possible for the pollution to not be dangerous)

\newpage
\item Tina will join a startup next year. Therefore,
Tina will join a startup next year or she will be unemployed.
\begin{align}
  P &= \text{``Tina will join a startup next year''}\\
  Q &= \text{``Tina will be unemployed''}\\
  P &\implies (P \lor Q)
\end{align}

$P \implies P$, and $(P \lor Q)$ is true if $P$ or $Q$ is true.
\begin{align}
\therefore P \implies (P \lor Q) \qed
\end{align}

\item If I work all night on this homework, I will answer all the
exercises. If I answer all the exercises, I will understand the
material. Therefore, if I work all night on this homework, I will
understand the material.

\begin{align}
  P &= \text{``I work all night on this homework''}\\
  Q &= \text{``I will answer all the exercises''}\\
  R &= \text{``I will understand the material''}\\
  (P &\implies Q) \land (Q \implies R) \implies P
\end{align}

$P \implies Q$, and $Q \implies R$.  So if $P$ is true, $Q$ is true, and if $Q$ is true then $R$ is true.
\begin{align}
\therefore P \implies R\qed
\end{align}
\end{enumerate}

\item
Recall that $\N=\{0,1,\ldots\}$ denotes the set of natural numbers,
and $\Z=\{\ldots,-1,0,1,\ldots\}$ denotes the set of integers.
\begin{enumerate}
\item Define $P(n)$ by
\[ P(n) = \forall m \in \N , \; m<n \implies
\neg (\exists k \in \N , \; n=mk \; \wedge \; k<n) \]
Concisely, for which numbers $n\in\N$ is $P(n)$ true?

$m | n \implies \exists k \in \N \ni n=mk$, which implies $\lnot P(n)$.  Therefore $P(n)$ is true when $m<n \implies m \nmid n$.  Such numbers are called prime numbers.
\begin{align}
\therefore P(n)\;\text{is true}\;\forall n \in \mathbb{P}\qed
\end{align}

\item Rewrite the following in a way that
removes all negations (``$\neg$, $\ne$'') but remains equivalent.
\[ \forall i . \; \neg \forall j . \;
\neg \exists k . \; 
(\neg \exists \ell . \; f(i,j) \ne g(k,\ell)). \]

$\forall i,j\;\exists k,l \ni f(i,j) = g(k,\ell)$

\item Prove or disprove:
$\forall m \in \Z . \; \exists n \in \Z . \; m \ge n$.

Let $n = m$.  By reflexion, $m \ge m$.
\begin{align}
\therefore \exists n \ni m \ge n
\end{align}

\item Prove or disprove:
$\exists m \in \Z . \; \forall n \in \Z . \; m \ge n$.

Assume such an $m$ exists.  Let $n = m+1$.  This implies
\begin{align}
  m &\ge m+1\\
  0 &\ge 1
\end{align}

This is false, so using proof by contradiction the proposition fails.

\end{enumerate}

\item
Alice and Bob are playing a game of chess,
with Alice to move first.
If $x_1,\dots,x_n$ represents a sequence of possible moves
(i.e., first Alice will make move $x_1$, then Bob will make move $x_2$,
and so on),
we let $W(x_1,\dots,x_n)$ denote the proposition that,
after this sequence of moves is completed,
Bob is checkmated.
\begin{enumerate}
\item State using quantifier notation the proposition that Alice
can force a checkmate on her second move, no matter how Bob plays.
\begin{align}
\forall x_1, x_2\;\exists x_3 \ni W(x_1, x_2, x_3)\;\text{is true}
\end{align}

\item Alice has many possibilities to choose from on her first move,
and wants to find one that lets her force a checkmate on her second move.
State using quantifier notation the proposition that $x_1$
is \emph{not} such a move.
\begin{align}
\forall x_2, x_3\; W(x_1, x_2, x_3)\;\text{is false}
\end{align}

\end{enumerate}

\item
Joan is either a knight or a knave.
Knights always tell the truth, and only the truth;
knaves always tell falsehoods, and only falsehoods.
Someone asks Joan, ``Are you a knight?''  She replies,
``If I am a knight then I'll eat my hat.''
\begin{enumerate}
\item Must Joan eat her hat?
\begin{align}
  P &= \text{``Joan is a knight''}\\
  Q &= \text{``Joan will eat hat''}\\
  P &\implies Q
\end{align}

If Joan is a knight, then she speaks truth, and so by the proposition will eat her hat.  If she is not a knight, then she speaks false, so we must negate her proposition:
\begin{align}
  \lnot (P &\implies Q) \iff (P \land \lnot Q)
\end{align}

So if Joan is a knave, then her statement must be false, but if it is false, then she must be a knight.  So Joan cannot be a knave, and thus must eat her hat.

\item Let's set this up as problem in propositional logic.
Introduce the following propositions:
\begin{eqnarray}
P &=& \text{``Joan is a knight''}\\
Q &=& \text{``Joan will eat her hat''}.
\end{eqnarray}
Translate what we're given into propositional logic,
i.e., re-write the premises in terms of these propositions.
\begin{align}
  P &\implies (P \implies Q) \iff (P \land Q)\\
  \lnot P &\implies \lnot (P \implies Q) \iff (P \land \lnot Q)
\end{align}

\item Using proof by enumeration,
prove that your answer from part (1) follows from the premises
you wrote in part (2).
(No inference rules allowed.)

\begin{tabular}{cccc}
  $P$ & $Q$ & $P \land Q$ & $P \land \lnot Q$ \\
  T & T &      T    &      F          \\
  T & F &      F    &      T          \\
  F & T &      F    &      F          \\
  F & F &      F    &      F          \\
\end{tabular}

So $P \implies (P \implies Q)$ is supported by enumeration, while $\lnot P \implies \lnot (P \implies Q)$ leads to contradiction.  Thus Joan will eat the hat.
\end{enumerate}

\item
For each claim below,
prove or disprove the claim.
\begin{enumerate}
\item Every positive integer can be expressed as the sum of two perfect squares. 
(A perfect square is the square of an integer. 0 may be used in the sum.)
\begin{align}
  a \in \mathbb{Z} &\implies \exists j,k \in \Z \ni a = j^2 + k^2\\
  (a = j^2 + k^2) &\implies |j|,|k| \le \sqrt{a}
\end{align}

This cannot be true for all positive integers, since it is not true for $3$; the only integers $\le \sqrt{3}$ are $0$ and $1$, and the max sum of their squares is $2$.
\begin{align}
  \therefore (j^2 + k^2 < 3)\;\forall j,k \le \sqrt{3}
\end{align}

\item For all rational numbers $a$ and $b$, $a^b$ is also rational.
\begin{align}
  b \in \mathbb{R} &\implies \exists j,k \in \Z \ni b = \frac{j}{k}\\
  k \nmid j &\implies a^b = a^{\frac{j}{k}} = a^j a^{\frac{1}{k}}\\
  |k| > 1 &\implies \exists a \ni a^{\frac{1}{k}} \notin \mathbb{R}\;(\text{e.g.}\;a \in \mathbb{P})\\
  \therefore \exists a, b &\in \mathbb{R} \ni a^b \notin \mathbb{R}
\end{align}

\end{enumerate}

\end{enumerate}
\end{document}
