\documentclass[11pt,fleqn]{article}

\usepackage{latexsym,epsf,fleqn}
\usepackage{amsmath,amsthm,amsfonts,amssymb}
\usepackage{mathtools}
\usepackage{array}
\usepackage{booktabs}

\usepackage{geometry}
\geometry{
  a4paper,
  total={170mm,257mm},
  left=20mm,
  top=20mm,
}

\newcommand\Set[2]{\{\,#1\mid#2\,\}}
\newcommand\underoverset[3]{\underset{#1}{\overset{#2}{#3}}}
\newcommand{\mbf}[1]{\mbox{{\bfseries #1}}}
\newcommand{\N}{\mathbb{N}}
\newcommand{\Z}{\mathbb{Z}}
\newcommand{\R}{\mathbb{R}}
\newcommand{\Q}{\mathbb{Q}}
\newcommand{\F}{\mathbb{F}}

\begin{document}

\section*{Exercises}

\subsection*{Section 6.9. Convolution Polynomial Rings}

\textbf{6.21.} Compute (by hand) the convolution polynomial product $\mathbf{c} = \mathbf{a} \star \mathbf{b}$ using the given value of $N$.
\begin{itemize}
\item[(a)] $N = 3$, \;\;\;$\mathbf{a}(x) = -1 + 4x + 5x^2, \;\;\; \mathbf{b}(x) = -1-3x-2x^2$ 
\begin{align}
    \mathbf{c} &= 1 + 3x + 2x^2 - 4x - 12x^2 - 8x^3 - 5x^2 - 15x^3 - 10x^4\nonumber\\
    &= 1 + 3x + 2x^2 - 4x - 12x^2 - 8 - 5x^2 - 15 - 10x\nonumber\\
    &= -22 - 11x + 15x^2\nonumber
\end{align}

\end{itemize}

\end{document}

